\documentclass[11pt,a4paper,twoside]{report}

%--------------------------------------------------
% global definitions to be used in the front page
%--------------------------------------------------
\providecommand{\coursename}{Progetto di Ingegneria Informatica}
\providecommand{\documentsubtitle}{Documentazione}
\providecommand{\annoacc}{2011-2012}
\providecommand{\principaladviser}{Prof. Andrea Bonarini}
\providecommand{\firstauthor}{Marcello Pogliani}
\providecommand{\firstauthorid}{742961}
\providecommand{\secondauthor}{Davide Tateo}
\providecommand{\secondauthorid}{743013}
\title{RoboTower}
\author{\firstauthor, \secondauthor}

\usepackage[margin=3cm]{geometry} % margins set here are for the TITLE page...

\usepackage[utf8]{inputenc}
\usepackage[italian]{babel}
%\usepackage{fancyhdr}
\usepackage{tikz}
\usetikzlibrary{arrows,positioning, decorations.text}

\usepackage{graphicx}
\usepackage{booktabs}
\usepackage{listings}
\lstset{columns=fullflexible}

\usepackage[T1]{fontenc}
\usepackage{palatino}
\usepackage{mathpazo} % font per caratteri matmatici simile a palatino
%\usepackage[scaled]{beramono}
\lstset{basicstyle=\ttfamily}

\usepackage[pdfauthor={\firstauthor, \secondauthor}, pdftitle={RoboTower}, colorlinks, linkcolor=black, urlcolor=black]{hyperref}

\usepackage{amsmath}
\usepackage{amsthm}
\newtheoremstyle{note} % name
	{\topsep} 	% Space above
	{\topsep} 	% Space below
	{\small}		% Body font
	{}		% Indent amount
	{\small\bfseries}% Theorem head font
	{:}		% Punctuation after theorem head
	{.5em}	% Space after theorem head
	{}		% Theorem head spec (can be left empty, meaning ‘normal’)
\theoremstyle{note}
\newtheorem*{nota}{Nota}	
%
\usepackage{settings/frontesp}

\usepackage{fancyhdr}
%% Cambia il carattere delle didascalie delle figure %%
\usepackage[font=small,format=plain,labelfont=bf,up,textfont=it,up]{caption}

\begin{document}

\titlep
\newgeometry{top=4cm,bottom=4cm,right=4cm,left=4cm}
\include{settings/theme}

% \pagenumbering{roman}
\tableofcontents
\cleardoublepage

% \pagenumbering{arabic}
% \setcounter{1}
%\pagestyle{fancy}



% --------------------------------------------------------------------------------------------------------------------
% Capitoli
\chapter{Introduzione}
%\markboth{Introduzione}{Introduzione}
\label{cap:introduzione}

Scopo del presente progetto è ideare e implementare un gioco appartenente alla categoria degli  ``HI-CoRG'' (Highly Interactive, Competitive Robogames). Si tratta di giochi che prevedono l'utilizzo di robot autonomi, e che contengano marcati elementi di interazione attiva tra giocatori umani e robot.

\emph{RoboTower} è un gioco strategico in tempo reale, basato sull'idea dei ``Tower defense'', in cui lo scopo del giocatore è difendere una ``torre'' dall'avanzata delle armate nemiche: in questo caso, è il giocatore umano che deve ostacolare l'avanzata del robot, ad esempio posizionando opportuni oggetti nel campo di gioco.
\chapter{Progetto del gioco}
%\markboth{Introduzione}{Introduzione}
\label{cap:progetto}

In questo capitolo verranno descritte le linee guida e i passi seguiti per progettare il gioco.

\section{Linee guida}
Innanzitutto, si è cercato di seguire alcune linee guida per lo sviluppo di un Robogame competitivo "di successo", ricavate a partire da alcuni articoli sull'argomento. Un Robogame deve avere alcune caratteristiche generali:
\begin{itemize}
\item consistere in almeno un robot e almeno un giocatore umano in grado di interagire tra di loro cooperativamente o competitivamente
\item basso costo e alta efficienza dell’uso dei componenti
\item sicuro da giocare
\item semplice e divertente (per chi ci gioca)
\item il robot deve essere visto dal giocatore come un agente razionale
\item il gioco deve essere provato sul campo
\item il gioco deve coinvolgere più sensi (del giocatore... e del robot) : vista, udito, tatto (ossia colori, suoni e forme)
\end{itemize}
In aggiunta a queste, il gioco che viene implementato deve avere un obiettivo chiaro e semplice, che deve poter essere suddiviso in sotto-obiettivi (punti) nel caso sia troppo lungo. La difficoltà del gioco deve essere adatta o adattabile al giocatore, le regole devono essere semplici da capire e imparare e le azioni facili da compiere. Inoltre, perché il gioco sia coinvolgente, tutto il sistema deve reagire reagisce prontamente alle azioni dell'utente.

L'efficacia dell'interazione con il giocatore viene influenzata notevolmente da robot, che dovrebbe reagire bene (a meno di semplificazioni) al comportamento umano, e avere alcune caratteristiche:
\begin{itemize}
\item Ricevere input nella maniera più affidabile e credibile possibile (anche a costo di semplificazioni)
\item Aspetto adatto al gioco
\item Comportamento che non appare casuale, ma razionale e pensato
\item Funzionamento in tempo reale
\end{itemize}

L'obiettivo generale della progettazione dei Robogame è quello di portare il gaming verso la sua naturale evoluzione verso la ``fisicità'', strada cominciata da Nintendo con il wii, e ormai accettata dalle maggiori case di videogiochi. I Robogames vogliono inoltre introdurre nella vita quotidiana il robot come qualcosa di ``familiare'' e utile, nonché diffondere l'interesse per la robotica ad un pubblico più ampio.

\section{Considerazioni preliminari al progetto del gioco}
Partendo dal presupposto che Robogame sia l’evoluzione del gaming tradizionale, abbiamo quindi cominciato a considerare i generi più usati nei videogiochi attuali:
\begin{itemize}
\item Strategici
\item Gestionali (esclusi in ottica robogame in quanto non adatti)
\item Sparatutto (esclusi in quanto fin troppo sfruttati, sia dai robot che dai videogiochi)
\item Giochi di ruolo (esclusi a causa del numero limitato di robot in gioco)
\item Platform (fisicamente poco realizzabili da un robot attuale)
\item Sportivi
\end{itemize}
Dopo aver escluso i generi non adatti, abbiamo considerato approfonditamente i due generi rimasti, e i loro limiti. Nel caso dei giochi sportivi, il limite è dato dalla necessità di avere un robot abbastanza dinamico, mentre i giochi strategici sono limitati dalla scarsa dinamicità di azione.

\paragraph{Uso della palla} Una prima distinzione all'interno dei giochi delle categorie considerate riguarda l'uso della palla. I giochi con palla rendono il robot più complesso, ma sono immediati e coinvolgenti, e non pongono gravi problemi legati all'introduzione di concetti complessi. Sono certamente semplici, intuitivi e dinamici. permettono di variare strategia, soprattutto se il gioco avviene con più agenti intelligenti in campo.
I giochi senza palla necessitano di robot in generale semplici, che richiedano al più velocità discrete. Sono meno coinvolgenti (anche se dipende dai gusti) e meno dinamici, o comunque se sono dinamici danno meno spazio a strategie pensate come ``razionali'' oppure paradossalmente prevedono l'introduzione di idee complesse, come quella di nascondiglio.
Un grosso problema legato all'uso della palla riguarda la velocità. Alcune idee per limitare la velocità della palla sono:
\begin{itemize}
\item Uso dei palloncini
\item Uso di palline da tennis sgonfie, con l’obbligo di rimbalzo (urto anelastico che “assorbe” energia cinetica)
\item Uso di zone determinate per il gioco e porte.
\end{itemize}

\paragraph{Giochi strategici} Dei giochi strategici abbiamo considerato:
\begin{itemize}
\item Derivati dai giochi da tavolo. Esempio: labirinto magico (famoso gioco da tavolo in scatola), in cui magari il labirinto è fatto con segnali luminosi (se possibile) o in altro modo. Il labirinto magico è un labirinto che può cambiare configurazione in cui bisogna trovare degli oggetti (oppure potrebbe essere l’uscita... oppure il robot. Altri esempi: giochi come “battaglia navale” o altri giochi di strategia pensati.
\item Tower Defense. Un esempio di Tower defense potrebbe essere un gioco ispirato a Rock Of Ages, che consiste nel difendere il proprio castello dall’assalto di una palla demolitrice e comandare la palla contro quello dell’avversario. In questo caso il robot potrebbe svolgere la funzione della “roccia”, evitando gli ostacoli insormontabili, e travolgendo/distruggendo (ovvero spostando... o passandoci sopra) quelli che invece potrebbero solo rallentarlo, o che gli bloccano il passaggio fino a raggiungere l’obiettivo. Più robot possono cooperare contro più umani per l’assalto al “castello”, oppure si possono fare due o più squadre robot/umano e simulare una battaglia alla “Rock Of Ages”
\item Da bambini e/o di intelligenza: giochi “infantili” possono essere trova \& nascondi (simili alla caccia al tesoro) distruggi \& costruisci, oppure un gioco in cui l’obiettivo del robot siano alcuni oggetti, però il robot non deve farsi individuare dal giocatore che ha gli oggetti... e il giocatore deve portare i robot a tradirsi attraverso questi oggetti, posizionandoli adeguatamente nell’area di gioco, sfruttando fondamentalmente lo stesso concetto della pesca.
\end{itemize}

\section{Design del gioco}
In particolare sono state stese tre bozze di gioco, una (RoboTower) basata sull'idea di Tower Defense, che è poi stata effettivamente implementata, e due bozze, entrambe focalizzate sull'uso della palla. Delle bozze scartate, la prima consiste in un gioco sportivo simile al tennis, in cui il robot deve svolgere azioni limitate rispetto all'umano, la seconda consiste in un gioco ispirato ai giochi infantili e basato su un palloncino.
In generale, è preferibile pensare a giochi in cui i ruoli di giocatore e robot siano diversi (come nella bozza Tower Defense), il che permette di aggirare eventuali limitazioni del robot, se non sfruttarle a vantaggio dell’esperienza di gioco.
Un altro problema, oltre a quello già citato della velocità, legato ai giochi che utilizzano la palla, è la necessità da parte del robot di effettuare movimenti complessi: la palla deve infatti essere sollevata da terra, colpita (eventualmente al volo) e direzionata, causando problemi nell'implementazione con una precisione accettabile su un robot di tali comportamenti. Per tale motivo, le bozze impieganti la palla sono state scartate.
Nello sviluppo dello storyboard di RoboTower, è stata prestata particolare attenzione a quali aspetti del gioco siano completamente controllabili dalla logica di gioco e quali invece non sono controllabili in maniera efficiente e\o sufficientemente precisa, lasciando il compito di rispettare queste ultime regole alla "buona fede" del giocatore.
\begin{itemize}
\item E’ stata accettata, almeno momentaneamente, l’idea del lancia-palle, usabile per abbattere sia la torre che le fabbriche. Il modo effettivo con cui queste strutture possono essere distrutte è ancora da discutere, e dipenderà dalle caratteristiche dello spara-palle.
\item Il gioco è stato progettato per essere sia facilmente espandibile, aumentando ad esempio il numero di robot o di torri da utilizzare, di conseguenza è possibile implementare versioni del gioco con più giocatori umani che collaborino con o contro più robot.
\end{itemize}

\chapter{Descrizione del gioco}
%\markboth{Introduzione}{Introduzione}
\label{cap:descrizione}

\section{Obiettivi}
Scopo del gioco è la difesa, da parte del giocatore, di un oggetto (la ``torre'') dall'assalto del (o dei) robot.

Per conseguire tale scopo, il giocatore può posizionare in maniera opportuna gli ostacoli a sua disposizione resistendo per un determinato tempo all'assalto del robot ed evitando che la torre venga distrutta.

\section{Requisiti}

\subsection*{Ambiente di gioco}
Non ci sono particolari vincoli sulla struttura del campo di gioco. Il gioco può pertanto svolgersi sia in un ambiente chiuso, purché sufficientemente grande da permettere il posizionamento degli ostacoli e il movimento del robot, che all'esterno, purché la superficie sia adatta al movimento del robot.

\subsection*{Destinatari}
Nella sua versione di base, il gioco si svolge tra un robot e un giocatore. È adatto a un target abbastanza vario, ed è rivolto a bambini e adulti tra 12 e i 60 anni.

RoboTower può essere facilmente esteso a più giocatori sia in senso cooperativo che in senso competitivo:
	\begin{itemize}
		\item È possibile introdurre diverse torri (e diversi set degli oggetti che verranno descritti in seguito), una per ogni giocatore. Ogni giocatore controlla la sua torre e tramite gli ostacoli cerca di allontanare il robot e ``convincerlo'' ad attaccare le altre torri (la strategia del robot potrebbe, ad esempio, cercare di attaccare la torre più vicina)
		\item Si possono introdurre diversi giocatori e/o robot che cooperano rispettivamente alla difesa e alla distruzione dell'unica torre
	\end{itemize}

\subsection*{Robot}
Il robot deve essere dotato di una telecamera per riconoscere gli ostacoli “fisici”, la torre e le trappole (costituite da marker a cui sono collegati contenuti concettuali). Sulla velocità non ci sono particolari requisiti, tuttavia deve essere sufficiente a rendere – in una certa misura – dinamico il gioco.

Il robot deve comunque disporre di un collegamento wireless per comunicare dati al computer e, nel caso non sia dotato di un processore sufficientemente potente per effettuare il riconoscimento delle immagini e gestire la logica di gioco, ricevere i comandi.

Per questi motivi, è possibile utilizzare sia robot commerciali, come lo Spykee della Meccano, che un robot appositamente costruito allo scopo.

Per rendere più coinvolgente e realistico il gioco, è preferibile - anche se non indispensabile - che il robot sia dotato di una o più armi (un lanciatore di palle, ad esempio) per colpire gli oggetti presenti nel gioco, come descritto in seguito.

\subsection*{Computer} Un computer svolge la funzione di arbitro e segnapunti. Viene utilizzato per dare inizio al gioco, calcolare i crediti del giocatore, visualizzare lo stato di attivazione delle trappole. Segna inoltre quanto manca alla fine del round, ossa il tempo che rimane al robot per raggiungere il suo obiettivo.
Infine il computer tiene traccia delle statistiche di gioco, come il numero di partite giocate, il numero di vittorie e sconfitte e i crediti di gioco totali del giocatore alla fine di ogni round

\section{Oggetti coinvolti}
Oltre al robot e al computer, già descritti nella sezione precedente, lo svolgimento del gioco prevede la presenza di altri oggetti: la torre, gli ostacoli, e le fabbriche.

\subsection*{Torre} La torre è un tubo di cartone di colore rosso uniforme. Il colore della torre deve essere differente da quello degli altri oggetti presenti nel campo di gioco, in modo che il robot possa riconoscerla tramite una telecamera, ed eventualmente anche calcolare la distanza a cui si trova. Inoltre la torre è dotata di un dispositivo di segnalazione radio, per indicare al robot l'avvenuto abbattimento della torre.

La torre dev'essere abbattuta dal robot. per farlo, il robot dovrà caricare la torre ed abbatterla.

\subsection*{Fabbriche} Le fabbriche sono oggetti di colore giallo uniforme che non sia simile a quello di altri oggetti del campo di gioco. Come nel caso della torre, le fabbriche devono essere riconosciute e possono essere distrutte dal robot. anche le fabbriche sono dotate di radiosegnale per indicare l'avvenuta distruzione al robot.

Ogni fabbrica (che viene posizionata all'inizio della partita) produce costantemente \emph{crediti di gioco}, finché non viene abbattuta dal robot.

\subsection*{Ostacoli} All'interno del gioco, ci sono due tipologie di ostacoli: gli ostacoli fisici e le trappole.
	\begin{itemize}
	\item Gli \emph{ostacoli fisici} rappresentano ostacoli insormontabili (come torrette difensive, barriere, montagne, mura). Sono una serie di oggetti a disposizione del giocatore che il robot non può spostare né abbattere, ma solo evitare. A differenza degli altri ostacoli, e come per le fabbriche, possono essere posizionati solo a inizio partita.
	\item Le \emph{trappole} sono ostacoli che modificano il comportamento del robot. Quando il robot passa sopra a questi ostacoli, a seconda della trappola in cui è incappato, svolge delle azioni:
		\begin{itemize}
		\item il robot viene obbligato a compiere un particolare movimento, ad esempio girare a destra o a sinistra, oppure tornare indietro (la direzione deve essere mantenuta per alcuni secondi)
		\item il robot viene bloccato completamente per un certo numero di secondi
		\item alcune funzionalità del robot vengono temporaneamente disabilitate o modificate per un certo periodo. Il robot può perdere la vista, diminuire la velocità o la capacità di cambiare direzione.
		\end{itemize}
	Questi ostacoli possono avere un effetto ben noto al giocatore, oppure possono avere effetti casuali, e anche negativi nei confronti del giocatore, come l'aumento di velocità del robot, oppure dell'energia del robot (ossia la durata massima del round).
	\end{itemize}

\section{Durata} 
Ogni round dura circa 5 minuti. può essere fermato il tempo in caso di problemi al robot, come ad esempio se il robot cade a causa di ostacoli non idonei sul campo di gioco.

Il gioco può terminare prima o dopo se la durata viene modificata da una trappola.

\section{Regole del gioco} 
Il gioco è strutturato in vari \emph{round}, ognuno dei quali può essere vinto dal robot o dal giocatore. Al termine della partita viene dichiarato vincitore chi, tra il robot e il giocatore, vince il maggior numero di round.

All'inizio del gioco, il robot è posizionato nella propria base. Gli ostacoli sono posizionati in un luogo (il ``deposito ostacoli'') posto, a discrezione del giocatore, a una certa distanza dall'area in cui verrà posizionata la torre, per rendere il gioco più (o meno) dinamico.


	\subsection*{Preparazione}
Il giocatore, prima dell'inizio di ogni round, posiziona la torre, poi dà avvio al gioco e posiziona gli ostacoli fisici e le fabbriche.

A seguito dell'avvio del gioco, dopo 30 secondi, il robot comincia a dirigersi verso la torre, cercando di evitare gli ostacoli.
	
	\subsection*{Svolgimento}
	Durante lo svolgimento del gioco, il giocatore può posizionare le trappole, cercando di evitare l'avanzata del robot. Il robot cerca di dirigersi verso la torre e abbatterla, evitando gli ostacoli fisici e rispettando gli ordini di quelli concettuali.

Il robot è dotato di una certa quantità di \emph{energia}. L'energia residua del robot è rappresentata dal tempo residuo del round, terminato il quale il robot si disattiva e non è più in grado di continuare, consegnando la vittoria al giocatore.

All'inizio del round il giocatore avrà a disposizione l'intero mazzo di carte trappola. Una volta attivata la trappola, essa viene disabilitata e sarà necessario atendere un tempo di rigenerazione, che dipende dalla quantità di fabbriche sul campo attive sul campo di gioco. Più fabbriche sono attive, più le trappole si rigeneranno velocemente.

	\subsection*{Conclusione} 
Il round termina quando il robot esaurisce tutta la sua energia oppure riesce ad abbattere la torre. In particolare, il giocatore vince il round quando il robot esaurisce tutta la sua energia, ossia finisce il tempo senza che venga abbattuta la torre, mentre il robot vince il round se riesce ad abbattere la torre prima del termine del tempo.
Al termine di ogni round gli ostacoli vengono riportati nel deposito ostacoli.
Inoltre i \emph{crediti di gioco} guadagnati andranno a sommarsi allo score del giocatore a fine partita, sia che il giocatore vinca la partita, sia che sia stato sconfitto dal robot.
A questo punto, il giocatore è pronto per iniziare un altro round.
\chapter{Architettura hardware}
%\markboth{Introduzione}{Introduzione}
\label{cap:architettura}

In questo capitolo viene descritta in dettaglio la componentistica hardware utilizzata per l'implementazione del gioco.

\section{Unità di elaborazione}
Sia l'elaborazione dei dati trasmessi dal robot mediante che la definizione dei comportamenti del robot in base alle logiche di gioco sono gestiti da un elaboratore esterno. Inoltre, l'elaboratore viene utilizzato dal giocatore per controllare lo stato del gioco (punteggio, tempo rimanente, ...) e per gestire l'avvio dello stesso.

\section{Il robot}

%TODO bisogna dire che per l'interpretazione dei pacchetti in arrivo è stato fatto reversing da un vecchio progetto etc etc etc?

L'attore principale del gioco è il robot autonomo \emph{Spykee}, un modello commerciale della Meccano \cite{spykeeweb}, disponibile nel laboratorio di Intelligenza Artificiale e Robotica (AIRLab) del Politecnico di Milano. Questo modello è già stato utilizzato con successo in precedenti progetti nell'ambito dei Robogame\footnote{in particolare nelle varie versioni di Robowii, l'ultima delle quali è descritta in \cite{robowii}}.

Spykee si muove tramite due cingoli (con tecnica differential drive) ed è dotato di una telecamera che permette di catturare immagini in formato \verb|JPEG| con una risoluzione di $320 \times 240$ pixel e a circa $20$ frame al secondo. Per comunicare con il computer che effettua il controllo, il robot crea all'accensione una rete \verb|Wi-Fi| ad-hoc.

\paragraph{Aggiunte hardware} A partire da quanto già realizzato in precedenti progetti, il robot è stato dotato di ulteriori sensori che lo rendono adatto all'utilizzo all'interno del gioco. Tutti i componenti hardware aggiunti al robot comunicano con l'unità di elaborazione tramite un collegamento wireless di tipo Zigbee (utilizzando un modulo XBee della Digi). Il protocollo Zigbee, è caratterizzato da una bassa velocità di trasmissione (comunque ampiamente sufficiente per gli scopi del progetto), ma da una buona facilità di utilizzo, specialmente in ambito embedded: viene infatti utilizzato come un collegamento seriale punto-punto tra i due dispositivi connessi.

Per interfacciare i vari dispositivi con il canale Zigbee è stata montata sul robot una scheda STM32F4 Discovery Board della ST, dotata di un microcontrollore ARM Cortex M4. Il firmware che permette di controllare l'hardware aggiunto al robot è stato sviluppato per il sistema operativo ChibiOS/RT\cite{chibios}.
L'utilizzo di un sistema operativo per microcontrollori permette di utilizzare astrazioni quali thread e mutex, nonché di astrarre l'hardware sottostante, consentendo quindi una certa modularità (isolando ogni funzione in un thread indipendente) e quindi un rapido sviluppo del firmware.

\begin{figure}[h]
\centering
\includegraphics[scale=0.4]{images/spykee}
\caption{Il robot Spykee utilizzato nel progetto}
\end{figure}

\paragraph{Sonar e led} Il robot è stato dotato di quattro sonar MaxSonar(R)-EZ della MaxBotix, uno per ognuno dei punti cardinali (nord, sud, ovest, est), che permettono di rilevare, e quindi evitare, eventuali ostacoli incontrati durante il movimento del robot. L'aggiunta di questo tipo di dispositivi è necessaria in quanto il robot non è progettato per muoversi autonomamente, ma soltanto per ricevere comandi manuali. Inoltre, Spykee è stato dotato di due strisce di LED (quattro LED gialli, quattro rossi e uno verde) che permettono di mostrare alcune informazioni relative allo stato del robot e/o del gioco nel complesso.

%TODO bisogna spiegare l'interfaccia del firmware? (comandi che riceve, pinout, ...)

%TODO sistemare!
Le varie schede elettroniche sono state raccolte in una apposita scatola, e sono alimentate dal pacco batteria di Spykee tramite un regolatore di tensione, che permette di ridurre la tensione proveniente dalla batteria (circa $9$ V) a quella di $5$ V. Un apposito interruttore, posto a valle dell'interruttore di alimentazione di Spykee, permette di spegnere o accendere le aggiunte hardware.

\section{Ostacoli attivi}
%TODO questa roba è tutta da sistemare!!!
Un problema che si è posto riguarda l'implementazione delle trappole (ostacoli che modificano il comportamento del robot). A questo scopo, inizialmente si sono valutate valutate alcune soluzioni basati su meccanismi di visione artificiale, in quanto all'interno del gioco viene già utilizzata una telecamera montata sul robot.

In particolare, sono stati presi in considerazione vari tipi di codici, tra cui i ``Datamatrix'' e i tag della libreria ``ARToolkitPlus''. Purtroppo, tutti questi meccanismi hanno dato scarsi risultati in termini di velocità oppure di qualità del riconoscimento (enorme dipendenza dalle condizioni di luce, scarsi risultati in movimento, lentezza degli algoritmi, ...). Queste caratteristiche li rendono inadatti per l'utilizzo che ne deve essere fatto all'interno del gioco, pertanto le soluzioni basate sul riconoscimento di tag sono state scartate.

Una soluzione che ha fornito prestazioni migliori, pur richiedendo l'aggiunta di ulteriore hardware al robot, è l'uso di tag RFID passivi (a 125 KHz). A seguito di alcune prove, si è riscontrato che questa tecnologia è dotata delle caratteristiche necessarie all'utilizzo all'interno del gioco. Per la lettura dei tag, è stato montato sul robot un lettore (l'ID-12 della ID-Innovations), che trasmette i dati mediante un collegamento seriale a 9600 bps all'evaluation board presente sul robot. Una limitazione di questo specifico modello di lettore è la presenza di un antenna interna, che limita pesantemente il posizionamento del lettore nel robot.

\section{Torri e fabbriche}
Le torri e le fabbriche sono state realizzate semplicemente mediante dei cilindri di cartoncino di colori differenti, di dimensione tele da poter essere rilevati ed abbattuti dal robot. In particolare, si è utilizzato un cilindro di colore rosso per la torre, e tre cilindri più piccoli di colore giallo per le fabbriche. 

La base dei cilindri contiene un interruttore, normalmente chiuso quando il cilindro è appoggiato a terra. L'interruttore viene aperto all'abbattimento della torre, e permette di inviare un segnale radio (utilizzando dei comuni trasmettitori del tipo di quelli utilizzati per l'apertura dei cancelli elettrici) a un ricevitore montato sul robot, e quindi di segnalare l'abbattimento di una torre o di una fabbrica all'unità di elaborazione.

Il ricevitore montato sul robot è il RX-4M-HCS dell'Aurel, ed è dotato di quattro canali separati con uscita open-drain (corrispondenti ai quattro pulsanti presenti sui corrispondenti trasmettitori, sempre della stessa marca). Pertanto, si sono impostati i relativi ingressi del microcontrollore come ingressi pull-up. Le uscite del ricevitore possono essere impostate, durante l'associazione con i trasmettitori, sia in modalità monostabile (ossia sono attive quando l'interruttore è chiuso) che in modalità bistabile (cambiano stato alla pressione dell'interruttore). Per la nostra applicazione si è scelto di utilizzare la modalità monostabile. 
\chapter{Architettura software}
\label{cap:architetturasw}

Per semplificare l'implementazione e agevolare il riuso del codice esistente, il software è stato realizzato utilizzando ROS (Robot Operating System) \cite{rosweb}. Si tratta di un middleware appositamente sviluppato per applicazioni robotiche, costituite da un elevato numero di moduli relativamente indipendenti che interagiscono tra loro.

Un sistema realizzato con ROS è suddiviso in un certo numero di \emph{nodi}, ovvero semplici processi\footnote{attualmente ROS supporta C++ e Python} indipendenti ed eseguiti in parallelo, che svolgono ciascuno una funzione elementare. Per migliorare l'organizzazione e la distribuzione dei sorgenti, i nodi possono poi essere raggruppati in \emph{package}. A loro volta, un gruppo di package può formare uno \emph{stack}.

ROS mette a disposizione due differenti tecniche con cui i nodi possono comunicare tra loro:
\begin{itemize}
 \item un paradigma ``publish-subscribe'': i nodi inviano (pubblicano) \emph{messaggi} su un certo \emph{argomento} (\emph{topic}), che vengono ricevuti dai nodi iscritti (subscribe), destinatari della comunicazione
 \item l'invocazione di \emph{servizi} esposti da altri nodi, secondo una semantica simile a quella di una chiamata a funzione
\end{itemize}

Oltre a funzionare come middleware di comunicazione interprocesso, ROS mette a disposizione alcuni programmi e librerie aggiuntive che svolgono funzioni utili allo sviluppo di applicazioni robotiche.

Il sistema realizzato risulta composto da vari nodi, ciascuno contenuto in un omonimo package all'interno della directory principale del progetto. I nodi, dettagliati nei paragrafi successivi, sono rappresentati nella figura \ref{fig:schemanodi}, insieme alle loro interfacce (messaggi scambiati e servizi esposti) e alle principali interazioni con le altre librerie utilizzate (esterne a ROS) e/o l'hardware.

\begin{figure}[h]
\resizebox{\linewidth}{!}{%
\begin{tikzpicture}[node distance=1cm, auto]
\tikzset{
   mynode/.style={rectangle,rounded corners,draw=black, top color=white, bottom color=white!50, thick, inner sep=1em, minimum size=3em, text centered},
   mynode2/.style={rectangle,rounded corners,draw=black, top color=white, bottom color=white!50, dashed, inner sep=1em, minimum size=1em, text centered},
}  
\node[mynode] (spykee) {\textbf{SpyKee}}; 
\node[mynode, below=4cm of spykee] (echoes) {\textbf{Echoes}}; 
\node[mynode, right=5cm of spykee] (vision) {\textbf{Vision}}; 
\node[mynode, below right=3cm of vision] (isaac) {\textbf{IsAac}}; 
\node[mynode2, below=1cm of isaac] (brian) {\emph{Mr. BRIAN}}; 
\node[mynode2, below left= 2cm of spykee] (robot) {\emph{Robot}}; 
\node[mynode2, above=1cm of vision] (opencv) {\emph{OpenCV + Blob Growing Algorithm}}; 
 
\draw[->, >=latex', shorten >=2pt, shorten <=2pt, bend left=10, thick] 
     (spykee.east) to node[auto, swap] {\ttfamily{spykee\_camera}}(vision.west); 

\draw[->, >=latex', shorten >=2pt, shorten <=2pt, bend left=10, thick] 
     (isaac.west) to node[auto, swap] {\ttfamily{spykee\_motion}}(spykee.south); 

\draw[->, >=latex', shorten >=2pt, shorten <=2pt, bend right=20, thick] 
     (echoes.east) to node[auto,swap] {\ttfamily{\begin{tabular}{c}
	 sonar\_data \\
     rfid\_data \\
     towers\_data
  \end{tabular}
}}(isaac.west); 
     
\draw[->, >=latex', shorten >=2pt, shorten <=2pt, bend right=10, thick, dashed] 
     (isaac.west) to node[auto, swap] {{\ttfamily{led\_data}} (servizio)}(echoes.east); 

\draw[->, >=latex', shorten >=2pt, shorten <=2pt, bend left=30, thick] 
     (vision.east) to node[auto, swap] {{\ttfamily{vision\_results}}}(isaac.north); 

\draw[<->, >=latex', shorten >=2pt, shorten <=2pt, bend left=0, thin, dashed] 
     (isaac.south) to node[auto, swap] {}(brian.north); 

\draw[<->, >=latex', shorten >=2pt, shorten <=2pt, bend left=0, thin, dashed] 
     (vision.north) to node[auto, swap] {}(opencv.south); 

\draw[<->, >=latex', shorten >=2pt, shorten <=2pt, bend left=7, thin, dashed] 
     (robot.north) to node[auto, swap] {Wi-fi}(spykee.west); 

\draw[<->, >=latex', shorten >=2pt, shorten <=2pt, bend right=7, thin, dashed] 
     (robot.south) to node[auto, swap] {Zigbee}(echoes.west); 

% The swap command corrects the placement of the text.

\end{tikzpicture} 
} 
\medskip

\caption{Struttura generale del sistema} 
\label{fig:schemanodi}
\end{figure}

\section{SpyKee}
Il nodo \emph{SpyKee} si occupa di interfacciare l'unità di elaborazione con le funzioni del robot che comunicano via Wi-Fi: permette di fornire i comandi ai cingoli e di ricevere le immagini catturate dalla telecamera. Questo nodo è frutto dell'adattamento a ROS di una libreria realizzata nei precedenti progetti analizzando mediante tramite un software di cattura dei pacchetti di rete la comunicazione tra il robot e il software di controllo fornito dalla Meccano.

SpyKee pubblica messaggi di tipo \verb|std_msgs::CompressedImage| sull'argomento \verb|spykee_camera|, che contengono le immagini ricevute dalla telecamera compresse in formato JPEG. Inoltre il nodo sottoscrive messaggi di tipo \verb|SpyKee::Motion| sull'argomento \verb|spykee_motion|, contenenti i comandi da inviare ai cingoli. Tali comandi sono costituiti da due interi compresi tra $-90$ e $90$, che rappresentano la velocità tangenziale e angolare del robot, e vengono convertiti dal nodo nei corrispondenti comandi al cingolo destro e sinistro.

\section{Echoes}
Il nodo \emph{Echoes} comunica via Zigbee con l'hardware aggiunto a posteriori al robot: sonar, led, lettore RFID, e ricevitori dei comandi inviati dagli interruttori posti sulle torri e sulle fabbriche.

Il nodo riceve i dati dalla porta seriale\footnote{di default viene utilizzato il device \texttt{/dev/ttyUSB0}, altrimenti il percorso del dispositivo deve essere fornito da riga di comando come primo argomento}, effettua il parsing delle stringhe ricevute, e pubblica i messaggi contenenti i dati rilevati:
\begin{itemize}
	\item \verb|Echoes::Sonar|, sull'argomento \verb|sonar_data|, contenenti i dati ricevuti dai sonar (valori delle distanze dei quattro sonar montati sul robot, espresse in millimetri).
	\item \verb|Echoes::Rfid|, sull'argomento \verb|rfid_data|, contenenti semplicemente una stringa identificativa del tag RFID rilevato
	\item \verb|Echoes::Tower| sull'argomento \verb|towers_data|, contenente un intero che corrisponde all'id della torre (o della fabbrica) abbattuta
\end{itemize}
Inoltre, il nodo espone un servizio di tipo \verb|Echoes::Led| che permette di controllare l'accensione dei led, e un altro servizio che consente di spegnere tutti i led. I led possono essere impostati nello stato di acceso, spento o lampeggiante. In particolare, per i led gialli e i led rossi, lo stato lampeggiante viene definito a livello dell'intero gruppo di led.

A causa dell'alto rumore presente nei valori provenienti dai sonar, il valore che viene pubblicato sul topic \verb|sonar_data| è filtrato attraverso un filtro a media mobile esponenziale. Questo significa che il dato pubblicato all'arrivo del $k$-esimo campione $x_{cur}$ registrato dal sonar è dato da %TODO sicuri che alpha = 0.3?!?
  \[ x_k = \alpha x_{cur} + (1 - \alpha) x_{k-1} \]
dove $\alpha = 0.3$, valore sufficientemente alto da rendere l'aggiornamento dei sonar abbastanza veloce per poter evitare efficacemente l'ostacolo, e sufficientemente basso da ridurre il rumore ad alta frequenza presente nel segnale.

\section{Vision: identificazione degli oggetti}

\emph{Vision} si occupa di analizzare le immagini provenienti dalla telecamera di Spykee, per rilevare la presenza di una torre o di una fabbrica. 

Riceve da \emph{SpyKee} i messaggi contenenti le immagini e, analizzata l'immagine, pubblica un messaggio di tipo \verb|Vision::Results| sull'argomento \verb|vision_results|. Questo messaggio contiene contenente i dati riguardanti gli oggetti trovati: in particolare, la posizione rispetto al centro dell'immagine, una stima della distanza in centimetri, e l'altezza e la larghezza del blob in pixel.

Il cuore del nodo è un algoritmo, già utilizzato nel progetto \cite{docmandelli}, che si occupa di identificare all'interno dell'immagine dei blob sufficientemente uniformi e di colore ``simile'' a quello degli oggetti che si stanno cercando. L'analisi viene effettuata basandosi su un algoritmo di tipo KNN, che necessita di una prima fase di addestramento. Al termine di questa fase, viene generato un classificatore (contenuto in un file \verb|.kcc|), tramite il quale l'algoritmo è in grado di associare ad ogni pixel dell'immagine una classe di identificazione. Le classi definite riguardano i pixel di colore simile a quello di una ``torre'' (classe R), oppure di una ``fabbrica'' (classe G). Una volta effettuata la classificazione dei pixel dell'immagine, viene eseguito un algoritmo per rilevare all'interno dell'immagine i blob di interesse, e quindi stabilire se all'interno dell'immagine è presente una torre o una fabbrica.

Il classificatore viene generato a partire da un file che contiene semplicemente un elenco di valori \verb|BGR| di pixel che si considerano del colore cercato. Per generare questo file è possibile utilizzare il nodo \emph{LittleEndian}: questo nodo carica le immagini da file oppure le riceve direttamente da \emph{SpyKee}, e consente di evidenziare nell'immagine le aree che corrispondono agli oggetti cercati, generando un file \verb|.dts| contenente l'insieme dei valori. A questo punto, per generare il classificatore, è sufficiente lanciare la prima volta \emph{Vision} indicando come parametro \verb|-L <filename.dts>|.

%TODO filtraggio etc?

\begin{nota}
È necessario prestare particolare attenzione nel training del classificatore, in quanto è una fase critica per il corretto funzionamento del gioco e il corretto riconoscimento degli oggetti. %% TODO TODO STA ROBA E' DA SISTEMARE!!!
\end{nota}

\section{RoboTower\_Game: la logica di gioco}
\begin{figure}
\centering
\includegraphics[scale=0.4]{images/rtgame}
\caption{Schermata principale dell'interfaccia di controllo del gioco}
\end{figure}

\emph{RoboTower\_Game} gestisce la logica ad ``alto livello'' del gioco e la comunicazione con l'utente tramite un'apposita interfaccia grafica, realizzata con le librerie Qt \cite{qtweb}. Si occupa di avviare e arrestare le partite, gestire lo stato del gioco, contare i punti, e tenere alcune semplici statistiche. Il nodo interagisce con gli altri processi:

\begin{itemize}
\item avviando e fermando il comportamento di basso livello del robot, a seconda dello stato corrente del gioco (mediante messaggi sull'argomento \verb|isaac_enable|)
\item ricevendo da \verb|Echoes| gli ID dei tag RFID letti e le informazioni riguardanti eventuali abbattimenti di torri o fabbriche
\item pubblicando messaggi relativi alle azioni speciali che devono essere eseguite, come spiegato nel seguito (argomento \verb|rfid_actions|), e invocando i servizi relativi all'accensione e allo spegnimento dei led relativi al punteggio
\end{itemize}

La maggior parte delle impostazioni di RoboTower\_Game possono essere configurate modificando sul file \verb|robotower.xml|, che si trova nella directory \verb|RoboTower_Game|. Un esempio di file di configurazione corretto è riportato in figura \ref{fig:configfile}\footnote{per semplicità sono stati omessi alcuni tag RFID e le relative azioni}. Il file di configurazione permette di specificare:
\begin{itemize}
\item I tempi del gioco (tag \lstinline|<time>|): la durata massima di un round del gioco (\lstinline|timetolive|), il tempo trascorso dall'inizio della partita (pressione del pulsante ``start'') e l'inizio del movimento del robot (\lstinline|setuptime|)
\item I parametri con cui viene calcolato il punteggio (tag \lstinline|<points>|): durante la partita, ogni secondo viene aggiunto un certo numero di punti per ogni torre e fabbrica che non sono stati ancora distrutti, definiti dagli attributi \lstinline|tower| e \lstinline|factory|
\item I parametri relativi a torre e fabbriche (tag \lstinline|<goals>|): in particolare, il numero di fabbriche (\lstinline|factories|) e l'id dell'obiettivo che dev'essere considerato come torre (\lstinline|towerid|)
\end{itemize}

\begin{figure}
{
\lstset{
  language=XML,
  morekeywords={encoding, robotower, config, time, points, goal, rfid, action, tag}}
\begin{lstlisting}
<robotower>
   <config>
      <time timetolive="300" setuptime="30" />
      <points tower="100" factory="30" />
      <goals towerid="4" factories="3" />
   </config>

   <rfid>
      <action name="lock_all">
         <tag id="4400F56CD1" num="1" />
         <tag id="4400F59195" num="2" />
         <tag id="4400F58B6D" num="3" />
      </action>
      
      <action name="disable_vision">
         <tag id="4400BDB1D9" num="4" />
         <tag id="4400BDC253" num="5" />
         <tag id="4B00DA3279" num="6" />
      </action>
   </rfid>
</robotower>
\end{lstlisting}
}
\caption{Il file di configurazione}
\label{fig:configfile}
\end{figure}

Il nodo si occupa inoltre di gestire i comportamenti ``speciali'' (action) del robot, attivati quando il robot si avvicina ai tag RFID cui sono associate. Per ogni trappola, è necessario specificare all'interno del blocco \lstinline|<action>| relativo all'azione associata, l'id e un numero (univoco) utilizzato per identificarlo nell'interfaccia grafica. Le azioni che possono essere specificate (parametri \lstinline|name|) sono:
\begin{itemize}
\item \lstinline|lock_all| blocca per 5 secondi i motori del robot
\item \lstinline|force_rotate| costringe per 5 secondi il robot a ruotare su se stesso, inibendo i comandi diretti a uno dei due cingoli
\item \lstinline|disable_vision| disabilita la visione del robot per 5 secondi
\item \lstinline|go_back| blocca per 5 secondi l'avanzamento del robot, costringendolo a tornare indietro
\item \lstinline|modify_time| modifica casualmente il tempo rimanente alla fine del round (sia in positivo che in negativo)
\end{itemize}
Una volta che viene letto un tag, questo viene disabilitato. I tag possono essere ricaricati, uno alla volta, dopo un periodo di tempo che dipende dal numero di fabbriche residue sul campo di gioco. %TODO RENDERLO CONFIGURABILE

\section{IsAac: il comportamento del robot}
\emph{IsAac} si occupa di controllare il comportamento di ``basso livello'' del robot durante il gioco. In base ai dati provenienti dai sensori, gestisce il comportamento del robot (i set-point per i cingoli e l'accensione di alcuni led).

Il nodo riceve i messaggi pubblicati da \emph{Echoes} e \emph{Vision}, e invia comandi a \emph{SpyKee} (messaggi \verb|Spykee::Motion| sull'argomento \verb|spykee_motion|) riguardanti il controllo dei cingoli. Inoltre invoca il servizio \verb|led_data| esposto da Echoes per il controllo dei led gialli e del led verde posti sul robot. I led gialli lampeggiano quando \emph{IsAac} è attivo, sono spenti quando non è attivo, e sono accesi fissi quando il robot è bloccato in una trappola. Il led verde lampeggia quando viene rilevata una torre o una fabbrica, altrimenti è spento.

Il nodo riceve da \emph{RoboTower\_Game} i comandi di attivazione e disattivazione, e le informazioni riguardanti il blocco del robot in una trappola associata ad un'azione che prevede la modifica dei comandi provenienti dai sensori oppure diretti agli attuatori (le altre azioni sono implementate direttamente in \emph{RoboTower\_Game}.

Per l'implementazione di questo nodo, è stata sfruttata la libreria Mr. Brian, sviluppata dal Politecnico di Milano all'interno di MRT \cite{mrt}. Questa libreria consente di definire il comportamento da applicare al robot come un insieme di regole scritte in logica fuzzy, utilizzando anche predicati che risultano dalla fuzzyficazione degli ingressi. La configurazione di Brian è basata su un insieme di file di testo: in questo modo è possibile modificare totalmente i comportamenti senza bisogno di ricompilare il codice sorgente. I file di configurazione, presenti nella cartella \verb|IsAac/config|, sono:
\begin{itemize}
\item \verb|behaviour.txt| contiene l'elenco, suddiviso per livelli, dei comportamenti (regole) definiti, ognuno dei quali è contenuto in un file \verb|.rul|
\item \verb|ctof.txt| associa ad ogni dato in ingresso (crisp) una membership function, che verrà utilizzata per la fase di fuzzyficazione. Le membership function sono definite nel file \verb|shape_ctof.txt|.
\item \verb|s_ftoc.txt| Definisce i valori in uscita e li associa alle funzioni definite nel file \verb|s_shape.txt|
\item \verb|Predicate.ini| Definisce i predicati fuzzy sulla base delle variabili fuzzy e/o di altri predicati
\item \verb|Cando.ini| Definisce le condizioni di attivazione per un determinato comportamento (le condizioni necessarie per cui eseguire quel comportamento sia sensato)
\item \verb|want.txt| Definisce le condizioni per cui è opportuno attivare un determinato comportamento
\end{itemize}
All'interno del manuale di MRT \cite{mrtmanual} è presente la descrizione completa del funzionamento di Mr. Brian e della sintassi dei file di configurazione.

I comportamenti che sono stati implementati riguardano:
\begin{itemize}
 \item Cercare casualmente la torre o le fabbriche (Search.rul)
 \item Raggiungere la torre o la fabbrica, una volta che è stata trovata (GoToTower.rul e GoToFactory.rul)
 \item Evitare gli ostacoli (AvoidObstacle.rul, KeepDistance.rul, GoBack.rul)
 \item Distruggere la torre o una fabbrica (Destroy.rul)
\end{itemize}

%TODO dettagliare meglio le regole o il loro funzionamento, spiegare la suddivisione in livelli???
\chapter{Conclusioni}
%\markboth{Introduzione}{Introduzione}
\label{cap:conclusioni}

Conclusioni... da scrivere alla fine ;) 

% Appendici for now don't needed!
\appendix
\chapter{Installazione del software}
%\markboth{Introduzione}{Introduzione}
\label{cap:installazione}

Il software realizzato è stato sviluppato per un sistema Linux. Le istruzioni seguenti si riferiscono principalmente alla distribuzione Ubuntu (in particolare, sono state testate con la versione 12.04) in quanto è l'unica supportata da ROS. Il software è stato testato anche con Debian: l'unica differenza risiede nell'installazione di ROS, che va fatta manualmente.

\section{Installazione del software lato PC}
\paragraph{Installazione di ROS} Le istruzioni di installazione di ROS sono presenti nel sito ufficiale, \url{www.ros.org}, e variano a seconda della distribuzione. In particolare, per Ubuntu, è necessario aggiungere i repository di ROS alle sorgenti software di sistema, dopodiché è possibile installare ROS direttamente dal package manager di Ubuntu. 

\begin{nota} I comandi per installare ROS su Ubuntu, riferiti all'ultima versione di ROS (\verb|fuerte|) e del sistema operativo (\verb|precise|) disponibile al momento della scrittura della documentazione, sono:
\begin{verbatim}
$ sudo sh -c `
    echo "deb http://packages.ros.org/ros/ubuntu precise main" 
    > /etc/apt/sources.list.d/ros-latest.list `
$ wget http://packages.ros.org/ros.key -O - | sudo apt-key add -
$ sudo apt-get update
$ sudo apt-get install ros-fuerte-desktop
\end{verbatim}
Una volta installati i pacchetti, è necessario fare in modo che le variabili di ambiente di ROS vengano impostate all'avvio:
\begin{verbatim}
$ echo "source /opt/ros/fuerte/setup.bash" >> ~/.bashrc
$ . ~/.bashrc
\end{verbatim}
Da ultimo, è necessario installare alcuni componenti aggiuntivi di ROS che verranno utilizzati in seguito:
\begin{verbatim}
$ sudo easy_install -U rosinstall vcstools rosdep
\end{verbatim}
\end{nota}

\paragraph{Installazione delle dipendenze} Per poter compilare correttamente il software realizzato, è necessario che siano presenti sul sistema, oltre a ROS, i seguenti pacchetti installabili dai repository della distribuzione Linux in uso:
\begin{itemize}
\item \verb|flex| e \verb|bison|, utilizzate per generare i parser di Mr. Brian
\item le librerie OpenCV (\verb|libopencv-dev|), utilizzate per elaborare le immagini in Vision e in LittleEndian
\item le librerie Qt4\footnote{Probabilmente le librerie Qt sono già installate da ROS come dipendenza} (\verb|libqt4-dev|), utilizzate per l'interfaccia grafica
\item le librerie Qt Mobility (\verb|qtmobility-dev|), che contengono QtMultimediaKit, utilizzato per la riproduzione dei suoni
\end{itemize}

È inoltre necessario installare il pacchetto di ROS \verb|ann|\footnote{\url{http://ros.org/wiki/ann}}, contenuto nello stack \texttt{ias\_perception}. Attualmente (marzo 2013), questo stack non è più disponibili sul sito di ROS (sono stati deprecati). Ann è parte del package "ias-perception", che si può scaricare dal repository git seguente:
\url{http://code.in.tum.de/git/ias-perception.git}, accessibile anche dall'interfaccia web che si trova al link \url{http://code.in.tum.de/indefero/index.php//p/ias-perception/source/tree/master/}. Una volta scaricato è sufficiente compilarlo con il comando \verb|make| sia dalla cartella principale \verb|ias-perception| che dalla sottocartella \verb|ann|, e successivamente copiarlo in una cartella presente nel PATH degli stack di ROS (ad esempio \verb|/opt/ros/fuerte/stacks|).

\paragraph{Compilazione del software} Innanzitutto, scaricare i sorgenti da git:
\begin{verbatim}
$ git clone git://github.com/pogliamarci/robotower.git
\end{verbatim}
oppure dalla pagina del progetto sull'airwiki in una cartella qualunque. Una volta scaricati i file, è necessario aggiungere la cartella del progetto alla variabile d'ambiente \verb|ROS_PACKAGE_PATH|, inserendo in fondo al file \verb|~/.bashrc| la riga
\begin{verbatim}
export ROS_PACKAGE_PATH=path:$ROS_PACKAGE_PATH
\end{verbatim}
dove \verb|path| va sostituito con il percorso completo della cartella che contiene i sorgenti del progetto. Dopo aver aggiornato il file \verb|~/.bashrc|, eseguire
\begin{verbatim}
$ source ~/.bashrc
\end{verbatim}
A questo punto, dovrebbe essere possibile compilare i sorgenti eseguendo semplicemente
\begin{verbatim}
$ make
\end{verbatim}
dalla cartella principale del progetto.

\begin{nota}
In alternativa, è possibile utilizzare il comando \verb|rosmake <nome_package>| per ognuno dei package realizzati. Quest'ultimo comando dovrebbe tentare di compilare e/o risolvere automaticamente le dipendenze, ma in generale risulta più lento e fornisce meno indicazioni sullo standard output rispetto all'utilizzo di \verb|make|.
\end{nota}

\section{Avvio del gioco} Una volta collegato al computer lo XBee e acceso il robot, collegarsi alla rete wireless creata da Spykee\footnote{il SSID è SPYKEE seguito da un codice identificativo del robot. Non è necessario inserire manualmente i parametri della connessione (indirizzo IP, ...), in quanto il robot funziona da server DHCP}. A questo punto, per avviare ROS e tutti i nodi che controllano il robot, è sufficiente eseguire il comando
\begin{verbatim}
$ roslaunch spykee.launch
\end{verbatim}
dalla cartella del progetto.

\begin{nota}
Se il nodo Echoes non riesce ad accedere alla porta seriale, è possibile che i permessi per l'utente corrente non consentano di aprire il dispositivo: per risolvere questo problema è sufficiente aggiungere l'utente corrente al gruppo \verb|dialout| oppure impostare i permessi corretti tramite regole di \verb|udev|.
\end{nota}

\section{Firmware del microcontrollore}

Il firmware è stato realizzato per la scheda (Evaluation Board) STM32F4 Discovery della ST. Nelle istruzioni che seguono, verrà utilizzato il programmatore integrato sulla scheda (ST-Link v2).

\subsection*{Prerequisiti}
\begin{enumerate}
\item Installare una toolchain GNU per ARM. Per la realizzazione del firmware, è stata utilizzata  CodeSourceryLite\footnote{scaricabile da \url{http://www.mentor.com/embedded-software/sourcery-tools/sourcery-codebench/editions/lite-edition/}}, che fornisce una versione precompilata della toolchain GNU (gcc, gdb, ...). È sufficiente scaricare l'archivio, decomprimerlo in una cartella, e aggiungere la sottocartella \verb|bin| dell'archivio nel \verb|PATH| dell’utente. Una volta che la toolchain è installata, il compilatore è raggiungibile dal comando 
\begin{verbatim}
$ arm-none-eabi-gcc
\end{verbatim}
Se si usa una toolchain diversa, potrebbe essere necessario cambiare i nomi degli eseguibili nel \verb|Makefile|.

\begin{nota}
Attualmente CodeSourceryLite viene rilasciato compilato per processori x86 a 32bit. Nel caso il sistema operativo in uso sia a 64 bit, è necessario installare le librerie \verb|ia32-libs|.
\end{nota}

\item Installare OpenOCD, scaricabile da \url{http://openocd.sourceforge.net}. Potrebbe essere necessario installare alcune dipendenze (autotools, automake, libusb ed altre), elencate comunque nella documentazione del programma.

\begin{nota}
Al momento della scrittura del presente manuale (luglio 2012), il programmatore ST-Link v2 è ufficialmente supportato solo sulla piattaforma Windows. Tuttavia, è presente un supporto sperimentale in OpenOCD: per poterlo utilizzare, è necessario scaricare e compilare la release di sviluppo dal repository git\footnote{\url{http://sourceforge.net/scm/?type=git&group_id=274635}}. Una volta scaricati i sorgenti, è sufficiente compilarli tramite i seguenti comandi  (dalla cartella in cui si è clonato il repository):
\begin{verbatim}
$ ./configure --enable-maintainer-mode --enable-stlink
$ make
# make install
\end{verbatim}
L'ultimo comando serve per installare OpenOCD nelle cartelle di sistema e configurare alcune regole di udev: dev'essere quindi eseguito con i privilegi di root.
\end{nota}

\item Scaricare i sorgenti di ChibiOS da \url{http://chibios.org}. Una volta scaricato e decompresso l'archivio, aprire il \verb|Makefile| presente nella cartella \verb|FirmwareSpykee| e modificare la riga
\begin{verbatim}
CHIBIOS = /opt/ChibiOS/ChibiOS_2.4.2
\end{verbatim}
inserendo il percorso corretto. Il firmware è stato testato con la versione 2.4.2 di ChibiOS.
\end{enumerate}

\subsection*{Compilazione ed installazione}

\paragraph{Compilazione} Posizionarsi nella cartella \verb|FirmwareSpykee| ed eseguire
\begin{verbatim}
$ make
\end{verbatim}
Se la compilazione ha successo, viene creato nella sottocartella \verb|build| il file \verb|ch.elf|.

\paragraph{Installazione} Per caricare il firmware sulla scheda, è necessario prima di tutto spegnere l'alimentazione\footnote{Attenzione! Collegare l'alimentazione della scheda quando è collegata al computer tramite USB potrebbe causare seri danni alla stessa e renderla inutilizzabile} (tramite l'apposito interruttore) e collegare la scheda al computer tramite USB. Si deve quindi avviare OpenOCD con il comando
\begin{verbatim}
$ openocd -f board/stm32f4discovery.cfg
\end{verbatim}
Quindi, mantenendo aperto OpenOCD, collegare il debugger gdb:
\begin{verbatim}
$ arm-none-eabi-gdb ch.elf
(gdb) target extended-remote localhost:3333
\end{verbatim}
dove \verb|ch.elf| è il percorso del firmware compilato che si vuole caricare sulla scheda. Una volta che \verb|gdb| è connesso ad OpenOCD, la seguente sequenza di comandi cancella il contenuto della memoria FLASH, carica il nuovo firmware, e quindi lo avvia:
\begin{verbatim}
(gdb) monitor reset halt
(gdb) monitor flash probe 0 
(gdb) monitor stm32f2x mass_erase 0 
(gdb) load 
(gdb) monitor reset halt 
(gdb) continue
\end{verbatim}

\begin{nota} A volte la scheda non viene riconosciuta da OpenOCD. Questo problema a volte viene risolto tenendo premuto il pulsante RESET sulla scheda mentre si avvia openOCD. Se anche così non funziona, è necessario cancellare il contenuto della memoria flash con l'utility ST Visual Programmer (disponibile solo per Windows).
\end{nota}

\section{Collegamenti hardware}
Attualmente, i pin presenti sulla Discovery Board montata sul robot sono assegnati in questo modo:
\begin{itemize}
\item Collegamento con i sonar (timer): \verb|PA8| (nord), \verb|PB4| (sud), \verb|PA0| (ovest), \verb|PC6| (est)
\item Seriale per collegamento con lo XBee, bitrate 115200 bps: \verb|PA2| (TX, va collegato al terminale \verb|RX| dello XBee) e \verb|PA3| (RX, va collegato al terminale \verb|TX| dello XBee)
\item Ricevitore dei segnali inviati
	\begin{itemize}
	\item Dalle fabbriche: \verb|PD0|, \verb|PD1| e \verb|PD2|
	\item Dalla torre: \verb|PD3|
	\end{itemize}
\item Led presenti sul robot:
	\begin{itemize}
	\item Led rossi: \verb|PE7| (0), \verb|PE8| (1), \verb|PE9| (2), \verb|PE10| (3)
	\item Led gialli: \verb|PE11| (0), \verb|PE12| (1), \verb|PE13| (3), \verb|PE14| (4)
	\item Led verde: \verb|PE15|
	\item Led a infrarossi (sulle spalle del robot): \verb|PD11|
	\end{itemize}
\item Lettore RFID (collegamento seriale a 9600 bps): \verb|PB11| (RX)
\end{itemize}
\begin{nota}
Il file \verb|board.h| permette di configurare le funzioni assegnate ai vari pin (input, output, alternate mode), secondo le informazioni presenti sul datasheet. Per modificare l'assegnamento dei pin, potrebb eessere necessario modificare - oltre a \verb|board.h| - il sorgente dei moduli del firmware che ne fanno uso.
\end{nota}

\paragraph{Alimentazione} Tutti i dispositivi (discovery board, XBee, lettore RFID, sonar, ricevitore Atmel, driver per i led a infrarossi) vanno alimentati a $+5$V DC tramite il regolatore di tensione collegato alla batteria principale di Spykee. La scheda contenente le resistenze necessarie al funzionamento dei led necessita soltanto del collegamento a massa.

\paragraph{Configurazione Zigbee} I dispositivi Xbee vengono utilizzati come un collegamento seriale punto-punto tra l'unità di elaborazione e il robot, pertanto per il debugging di problemi di configurazione è sufficiente utilizzare un terminale seriale, come \verb|screen| o \verb|minicom| su Linux. Per configurare i dispositivi Xbee è necessario utilizzare il programma X-CTU scaricabile dal sito Digi. Una volta aperta la scheda di configurazione, è sufficiente impostare su entrambi i dispositivi il bitrate di $115200$ bps, e impostare come Destination Address (High e Low) il Serial Number dell'altro Xbee.

\section{Comandi per il firmware del robot}
Il firmware installato sulla discovery board presente nel robot accetta alcuni comandi che vengono inviati dal nodo \verb|Echoes| attraverso il collegamento seriale, e permettono di controllare i led:
\begin{itemize}
\item \verb|reset|: spegne tutti i led
\item \verb|led|: comanda i led colorati. Il comando è seguito da \verb|R|, \verb|Y| oppure \verb|G| a seconda che si vogliano comandare rispettivamente i led rossi, i led gialli oppure il led verde che c'è sulla testa di Spykee. Il secondo argomento del comando può essere \verb|B| per attivare la modalità lampeggiante del gruppo di led selezionato, oppure una stringa di cifre binarie che indicano se ognuno dei led del gruppo deve essere spento (\verb|0|) oppure acceso (\verb|1|). Ad esempio, il comando
\begin{verbatim}
    led R 1101
\end{verbatim}
accende il primo, il secondo e l'ultimo dei led rossi e spegne il terzo, mentre il comando
\begin{verbatim}
    led G B
\end{verbatim}
fa lampeggiare il led verde sulla testa del robot.
\item \verb|infrared on| accende i led a infrarossi, mentre \verb|infrared off| li spegne.
\end{itemize}

I messaggi inviati dal robot al computer attraverso il collegamento seriale sono composti da una singola linea (sono terminati da \verb|CR LF|) e contengono all'inizio la tipologia tra parentesi quadre. Le tipologie di messaggio che sono state definite, insieme con un esempio del loro formato, sono:
\begin{itemize}
\item \verb|[SONAR] N:1443,S:161,W:3444,E:3458| contiene per i quattro sonar installati (nord, sud, ovest, est) la distanza rilevata in millimetri
\item \verb|[RFID] <id>| contiene l'ID del tag RFID rilevato. Contiene anche il checksum, calcolato come da datasheet del lettore ID-12 (la correttezza viene calcolata da Echoes e non a bordo del firmware).
\item \verb|[TOWER] destroyed <N>| indica che una torre o fabbrica è stata distrutta. \verb|<N>| è compreso tra $1$ e $4$ e identifica la torre o la fabbrica che è stata distrutta.
\end{itemize}
% \include con i nomi dei capitoli

% --------------------------------------------------------------------------------------------------------------------
% Bibliografia - nothing for now!
\bibliographystyle{plain}
\begin{thebibliography}{9}

\bibitem{rosweb}
  ROS Wiki, \url{www.ros.org}

\bibitem{spykeeweb}
  Meccano, Spykee: the Spy Robot, \url{http://www.spykeeworld.com}

\bibitem{chibios}
  The ChibiOS/RT Project, \url{http://chibios.org}

\bibitem{qtweb}
	Qt - Cross-platform application and UI framework, \url{http://qt.nokia.com}

\bibitem{robowii}
  ROBOWII, \url{http://airlab.elet.polimi.it/index.php/ROBOWII}

\bibitem{mrtmanual}
  MRT: Modular Robotics Toolkit, \url{http://airlab.elet.polimi.it/index.php/MRT}

\bibitem{mrt}
  A. Bonarini, M. Matteucci, M. Restelli, \emph{MRT: Robotics Off-the-Shelf with the Modular Robotic Toolkit},
  in D. Brugali (ed.), \emph{Software Engineering for Experimental Robotics}, STAR 30, Berlin, Springer-Verlag, 2007, 345-364

\bibitem{docmandelli}
  C. Mandelli, D. Zamponi, \emph{RoboWII 2.1: Implementazione di un gioco robotico}, \url{http://airlab.elet.polimi.it/images/1/1f/RoboWii_2.1-ZamponiMandelli.pdf}

\bibitem{brian}
  A. Bonarini, M. Matteucci, M. Restelli, \emph{Concepts and fuzzy models for behavior-based robotics}, International Journal of Approximate Reasoning \textbf{41} (2006), 110-127

\bibitem{st}
  st

\bibitem{digi}
  digi
  
\bibitem{maxbotix}
  max
  
\bibitem{idinnovations}
  id
  
\bibitem{aurel}
  aurel
    

\end{thebibliography}

% --------------------------------------------------------------------------------------------------------------------

\end{document}
