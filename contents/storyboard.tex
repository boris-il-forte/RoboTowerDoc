\chapter{Descrizione del gioco}
%\markboth{Introduzione}{Introduzione}
\label{cap:descrizione}

\section{Obiettivi}
Scopo del gioco è la difesa, da parte del giocatore, di un oggetto (la ``torre'') dall'assalto del (o dei) robot.

Per conseguire tale scopo, il giocatore può posizionare in maniera opportuna gli ostacoli a sua disposizione resistendo per un determinato tempo all'assalto del robot ed evitando che la torre venga distrutta.

\section{Requisiti}

\subsection*{Ambiente di gioco}
Non ci sono particolari vincoli sulla struttura del campo di gioco. Il gioco può pertanto svolgersi sia in un ambiente chiuso, purché sufficientemente grande da permettere il posizionamento degli ostacoli e il movimento del robot, che all'esterno, purché la superficie sia adatta al movimento del robot.

\subsection*{Destinatari}
Nella sua versione di base, il gioco si svolge tra un robot e un giocatore. È adatto a un target abbastanza vario, ed è rivolto a bambini e adulti tra 12 e i 60 anni.

RoboTower può essere facilmente esteso a più giocatori sia in senso cooperativo che in senso competitivo:
	\begin{itemize}
		\item È possibile introdurre diverse torri (e diversi set degli oggetti che verranno descritti in seguito), una per ogni giocatore. Ogni giocatore controlla la sua torre e tramite gli ostacoli cerca di allontanare il robot e ``convincerlo'' ad attaccare le altre torri (la strategia del robot potrebbe, ad esempio, cercare di attaccare la torre più vicina)
		\item Si possono introdurre diversi giocatori e/o robot che cooperano rispettivamente alla difesa e alla distruzione dell'unica torre
	\end{itemize}

\subsection*{Robot}
Il robot deve essere dotato di una telecamera per riconoscere gli ostacoli “fisici”, la torre e le trappole (costituite da marker a cui sono collegati contenuti concettuali). Sulla velocità non ci sono particolari requisiti, tuttavia deve essere sufficiente a rendere – in una certa misura – dinamico il gioco.

Il robot deve comunque disporre di un collegamento wireless per comunicare dati al computer e, nel caso non sia dotato di un processore sufficientemente potente per effettuare il riconoscimento delle immagini e gestire la logica di gioco, ricevere i comandi.

Per questi motivi, è possibile utilizzare sia robot commerciali, come lo Spykee della Meccano, che un robot appositamente costruito allo scopo.

Per rendere più coinvolgente e realistico il gioco, è preferibile - anche se non indispensabile - che il robot sia dotato di una o più armi (un lanciatore di palle, ad esempio) per colpire gli oggetti presenti nel gioco, come descritto in seguito.

\subsection*{Computer} Un computer svolge la funzione di arbitro e segnapunti. Viene utilizzato per dare inizio al gioco, calcolare i crediti del giocatore, visualizzare lo stato di attivazione delle trappole. Segna inoltre quanto manca alla fine del round, ossa il tempo che rimane al robot per raggiungere il suo obiettivo.
Infine il computer tiene traccia delle statistiche di gioco, come il numero di partite giocate, il numero di vittorie e sconfitte e i crediti di gioco totali del giocatore alla fine di ogni round

\section{Oggetti coinvolti}
Oltre al robot e al computer, già descritti nella sezione precedente, lo svolgimento del gioco prevede la presenza di altri oggetti: la torre, gli ostacoli, e le fabbriche.

\subsection*{Torre} La torre è un tubo di cartone di colore rosso uniforme. Il colore della torre deve essere differente da quello degli altri oggetti presenti nel campo di gioco, in modo che il robot possa riconoscerla tramite una telecamera, ed eventualmente anche calcolare la distanza a cui si trova. Inoltre la torre è dotata di un dispositivo di segnalazione radio, per indicare al robot l'avvenuto abbattimento della torre.

La torre dev'essere abbattuta dal robot. per farlo, il robot dovrà caricare la torre ed abbatterla.

\subsection*{Fabbriche} Le fabbriche sono oggetti di colore giallo uniforme che non sia simile a quello di altri oggetti del campo di gioco. Come nel caso della torre, le fabbriche devono essere riconosciute e possono essere distrutte dal robot. anche le fabbriche sono dotate di radiosegnale per indicare l'avvenuta distruzione al robot.

Ogni fabbrica (che viene posizionata all'inizio della partita) produce costantemente \emph{crediti di gioco}, finché non viene abbattuta dal robot.

\subsection*{Ostacoli} All'interno del gioco, ci sono due tipologie di ostacoli: gli ostacoli fisici e le trappole.
	\begin{itemize}
	\item Gli \emph{ostacoli fisici} rappresentano ostacoli insormontabili (come torrette difensive, barriere, montagne, mura). Sono una serie di oggetti a disposizione del giocatore che il robot non può spostare né abbattere, ma solo evitare. A differenza degli altri ostacoli, e come per le fabbriche, possono essere posizionati solo a inizio partita.
	\item Le \emph{trappole} sono ostacoli che modificano il comportamento del robot. Quando il robot passa sopra a questi ostacoli, a seconda della trappola in cui è incappato, svolge delle azioni:
		\begin{itemize}
		\item il robot viene obbligato a compiere un particolare movimento, ad esempio girare a destra o a sinistra, oppure tornare indietro (la direzione deve essere mantenuta per alcuni secondi)
		\item il robot viene bloccato completamente per un certo numero di secondi
		\item alcune funzionalità del robot vengono temporaneamente disabilitate o modificate per un certo periodo. Il robot può perdere la vista, diminuire la velocità o la capacità di cambiare direzione.
		\end{itemize}
	Questi ostacoli possono avere un effetto ben noto al giocatore, oppure possono avere effetti casuali, e anche negativi nei confronti del giocatore, come l'aumento di velocità del robot, oppure dell'energia del robot (ossia la durata massima del round).
	\end{itemize}

\section{Durata} 
Ogni round dura circa 5 minuti. può essere fermato il tempo in caso di problemi al robot, come ad esempio se il robot cade a causa di ostacoli non idonei sul campo di gioco.

Il gioco può terminare prima o dopo se la durata viene modificata da una trappola.

\section{Regole del gioco} 
Il gioco è strutturato in vari \emph{round}, ognuno dei quali può essere vinto dal robot o dal giocatore. Al termine della partita viene dichiarato vincitore chi, tra il robot e il giocatore, vince il maggior numero di round.

All'inizio del gioco, il robot è posizionato nella propria base. Gli ostacoli sono posizionati in un luogo (il ``deposito ostacoli'') posto, a discrezione del giocatore, a una certa distanza dall'area in cui verrà posizionata la torre, per rendere il gioco più (o meno) dinamico.


	\subsection*{Preparazione}
Il giocatore, prima dell'inizio di ogni round, posiziona la torre, poi dà avvio al gioco e posiziona gli ostacoli fisici e le fabbriche.

A seguito dell'avvio del gioco, dopo 30 secondi, il robot comincia a dirigersi verso la torre, cercando di evitare gli ostacoli.
	
	\subsection*{Svolgimento}
	Durante lo svolgimento del gioco, il giocatore può posizionare le trappole, cercando di evitare l'avanzata del robot. Il robot cerca di dirigersi verso la torre e abbatterla, evitando gli ostacoli fisici e rispettando gli ordini di quelli concettuali.

Il robot è dotato di una certa quantità di \emph{energia}. L'energia residua del robot è rappresentata dal tempo residuo del round, terminato il quale il robot si disattiva e non è più in grado di continuare, consegnando la vittoria al giocatore.

All'inizio del round il giocatore avrà a disposizione l'intero mazzo di carte trappola. Una volta attivata la trappola, essa viene disabilitata e sarà necessario atendere un tempo di rigenerazione, che dipende dalla quantità di fabbriche sul campo attive sul campo di gioco. Più fabbriche sono attive, più le trappole si rigeneranno velocemente.

	\subsection*{Conclusione} 
Il round termina quando il robot esaurisce tutta la sua energia oppure riesce ad abbattere la torre. In particolare, il giocatore vince il round quando il robot esaurisce tutta la sua energia, ossia finisce il tempo senza che venga abbattuta la torre, mentre il robot vince il round se riesce ad abbattere la torre prima del termine del tempo.
Al termine di ogni round gli ostacoli vengono riportati nel deposito ostacoli.
Inoltre i \emph{crediti di gioco} guadagnati andranno a sommarsi allo score del giocatore a fine partita, sia che il giocatore vinca la partita, sia che sia stato sconfitto dal robot.
A questo punto, il giocatore è pronto per iniziare un altro round.