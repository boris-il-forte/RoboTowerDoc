\chapter{Scenari di gioco}
%\markboth{Introduzione}{Introduzione}
\label{cap:scenari}

In questo capitolo viene riassunto lo schema generale di un round del gioco, specificati ulteriori dettagli del funzionamento, e presentati alcuni scenari d'uso.

\begin{enumerate}
    \item Il giocatore posiziona il robot e la torre nella stanza
	\item Il giocatore sceglie gli ostacoli fisici da comprare col credito iniziale e da avvio al gioco
	\item Il giocatore piazza gli ostacoli fisici fino a che un segnale lo avvisa che la partita comincia.
	\item Il robot si mette in movimento verso la torre nel tentativo di abbatterla.
	\item Il giocatore può acquistare una trappola e posizionarla se il suo credito è sufficiente. Può posizionare la trappola dove vuole, a condizione di prendere una trappola per volta e di piazzarla a più di 50 cm dal robot.
	\item Il robot che cada in una trappola, subirà gli effetti imposti da essa, che possono essere:
		\begin{enumerate}
		\item Perdere energia vitale (si accorcia il suo “time to live”)
		\item Aumentare la sua energia vitale (si allunga il suo “time to live”)
		\item Venire immobilizzato per alcuni secondi
		\item Non potere sparare per alcuni secondi
		\item Non potere vedere nulla nel suo campo visivo per alcuni secondi
		\item Diminuire la velocità per alcuni secondi
		\item Aumentare la velocità per alcuni secondi
		\item Diminuire la sua capacità rotatoria per alcuni secondi
		\item Essere obbligato a cambiare direzione per alcuni secondi
		\end{enumerate}
	\item Il robot può, se lo ritiene opportuno, distruggere le fabbriche, e diminuire la produzione di crediti del giocatore.
	\item Se il robot distrugge la torre, ha vinto il round.
	\item Se il giocatore resiste all’assalto del robot fino a far consumare la sua energia vitale completamente (il suo “Time to live”) vince il round.
	\item Se il robot si arrende per impossibilità fisica di rimettersi in condizioni di operare, il giocatore vince il round.
	\end{enumerate}

Alcune regole fondamentali del gioco risultano impossibili da verificare da parte della logica di gioco, o quantomeno risultano impossibili da verificare in maniera efficiente. Il rispetto di queste regole da parte del giocatore è pertanto basato sulla fiducia nei suoi confronti:
	\begin{itemize}
	\item Il giocatore non può colpire il robot.
	\item Il giocatore non può posizionare le trappole troppo vicino al robot
	\item Il giocatore non può chiudere, con le trappole fisiche, ogni percorso che il robot potrebbe seguire per raggiungere la torre.
	\item Il giocatore non può spostare trappole fisiche posizionate in precedenza, ne posizionare trappole fisiche una volta che il robot sia in movimento.
	\item Il giocatore non può frapporsi tra il robot e il suo obbiettivo.
	\item Il giocatore non può prendere dal “deposito ostacoli” più trappole alla volta.
	\item Il giocatore non può prendere trappole senza comprarle.
	\item Il giocatore non può spostare le trappole che ha posizionato.
	\end{itemize}