\chapter{Architettura software}
\label{cap:architetturasw}

Per semplificare l'implementazione del sistema e agevolare il riuso del codice esistente, è stato utilizzato ROS (Robot Operating System) \cite{rosweb}. Si tratta di un middleware appositamente sviluppato per applicazioni robotiche costituite da un elevato numero di moduli relativamente indipendenti e interagenti tra loro.

Un sistema realizzato con ROS è suddiviso in un certo numero di \emph{nodi}, ovvero semplici processi\footnote{attualmente ROS supporta C++ e Python} indipendenti ed eseguiti in parallelo, che svolgono ciascuno una funzione elementare. Per migliorare l'organizzazione e la distribuzione dei sorgenti, i nodi possono poi essere raggruppati in \emph{package}. A loro volta, un gruppo di package può formare uno \emph{stack}.

Le due differenti metodologie che mette a disposizione l'infrastruttura di ROS per permettere ai diversi processi di comunicare sono:
\begin{itemize}
 \item un paradigma ``publish-subscribe'', in cui i nodi inviano (pubblicano) \emph{messaggi} su un certo \emph{argomento} (\emph{topic}), a cui i nodi destinatari della comunicazione si iscrivono
 \item l'invocazioen di \emph{servizi} esposti da altri nodi, secondo una semantica simile a quella di una chiamata a funzione
\end{itemize}

Oltre a funzionare come middleware di comunicazione interprocesso, ROS mette a disposizione alcuni programmi e librerie aggiuntive che svolgono funzioni utili allo sviluppo di applicazioni robotiche.

Il sistema da noi realizzato risulta composto da vari nodi, ciascuno contenuto in un package omonimo all'interno della directory principale del progetto. I nodi sono rappresentati nella figura \ref{fig:schemanodi}, insieme alle loro interfacce (messaggi scambiati e servizi esposti) e alle principali interazioni con le altre librerie utilizzate (esterne a ROS) e/o l'hardware. Nei paragrafi successivi verranno presentate le funzioni svolte dai singoli nodi.

\begin{figure}[h]
\resizebox{\linewidth}{!}{%
\begin{tikzpicture}[node distance=1cm, auto]
\tikzset{
   mynode/.style={rectangle,rounded corners,draw=black, top color=white, bottom color=white!50, thick, inner sep=1em, minimum size=3em, text centered},
   mynode2/.style={rectangle,rounded corners,draw=black, top color=white, bottom color=white!50, dashed, inner sep=1em, minimum size=1em, text centered},
}  
\node[mynode] (spykee) {\textbf{SpyKee}}; 
\node[mynode, below=4cm of spykee] (echoes) {\textbf{Echoes}}; 
\node[mynode, right=5cm of spykee] (vision) {\textbf{Vision}}; 
\node[mynode, below right=3cm of vision] (isaac) {\textbf{IsAac}}; 
\node[mynode2, below=1cm of isaac] (brian) {\emph{Mr. BRIAN}}; 
\node[mynode2, below left= 2cm of spykee] (robot) {\emph{Robot}}; 
\node[mynode2, above=1cm of vision] (opencv) {\emph{OpenCV + Blob Growing Algorithm}}; 
 
\draw[->, >=latex', shorten >=2pt, shorten <=2pt, bend left=10, thick] 
     (spykee.east) to node[auto, swap] {\ttfamily{spykee\_camera}}(vision.west); 

\draw[->, >=latex', shorten >=2pt, shorten <=2pt, bend left=10, thick] 
     (isaac.west) to node[auto, swap] {\ttfamily{spykee\_motion}}(spykee.south); 

\draw[->, >=latex', shorten >=2pt, shorten <=2pt, bend right=20, thick] 
     (echoes.east) to node[auto,swap] {\ttfamily{\begin{tabular}{c}
	 sonar\_data \\
     rfid\_data \\
     towers\_data
  \end{tabular}
}}(isaac.west); 
     
\draw[->, >=latex', shorten >=2pt, shorten <=2pt, bend right=10, thick, dashed] 
     (isaac.west) to node[auto, swap] {{\ttfamily{led\_data}} (servizio)}(echoes.east); 

\draw[->, >=latex', shorten >=2pt, shorten <=2pt, bend left=30, thick] 
     (vision.east) to node[auto, swap] {{\ttfamily{vision\_results}}}(isaac.north); 

\draw[<->, >=latex', shorten >=2pt, shorten <=2pt, bend left=0, thin, dashed] 
     (isaac.south) to node[auto, swap] {}(brian.north); 

\draw[<->, >=latex', shorten >=2pt, shorten <=2pt, bend left=0, thin, dashed] 
     (vision.north) to node[auto, swap] {}(opencv.south); 

\draw[<->, >=latex', shorten >=2pt, shorten <=2pt, bend left=7, thin, dashed] 
     (robot.north) to node[auto, swap] {Wi-fi}(spykee.west); 

\draw[<->, >=latex', shorten >=2pt, shorten <=2pt, bend right=7, thin, dashed] 
     (robot.south) to node[auto, swap] {Zigbee}(echoes.west); 

% The swap command corrects the placement of the text.

\end{tikzpicture} 
} 
\medskip

\caption{Struttura generale del sistema} 
\label{fig:schemanodi}
\end{figure}

%TODO È IL CASO DI METTERE QUI LE DESCRIZIONI COMPLETE DEI MESSAGGI??? In generale, come la organizziamo questa parte?

\section{SpyKee}
Il nodo \emph{SpyKee} controlla le comunicazioni che avvengono tra unità di elaborazione e robot via Wi-Fi: i comandi forniti ai cingoli e le immagini catturate dalla telecamera. Questo nodo è l'adattamento a ROS della libreria realizzata nei precedenti progetti mediante l'analisi (tramite un software di sniffing di rete) dei pacchetti scambiati tra il robot e il software di controllo fornito dalla Meccano.

SpyKee pubblica messaggi di tipo \verb|std_msgs::CompressedImage| sull'argomento \verb|spykee_camera|, che contengono le immagini ricevute dalla telecamera, compresse in formato JPEG. Inoltre il nodo sottoscrive messaggi di tipo \verb|SpyKee::Motion| sull'argomento \verb|spykee_motion|, contenenti i comandi da inviare ai cingoli. Tali comandi costituiti da due interi compresi tra $-90$ e $90$ che rappresentano la velocità tangenziale e angolare del robot, e che vengono convertiti dal nodo nei corrispondenti comandi al cingolo destro e sinistro.

\section{Echoes}
\emph{Echoes} comunica via Zigbee con l'hardware aggiunto a posteriori al robot: sonar, led, lettore RFID, ricevitori dei comandi inviati dagli interruttori posti sulle torri e sulle fabbriche.

Il nodo riceve i dati dalla porta seriale\footnote{di default da \texttt{/dev/ttyUSB0}, nel caso il dispositivo abbia un percorso differente questo dev'essere specificato da riga di comando come primo argomento}, effettua il parsing delle stringhe ricevute, e pubblica messaggi contenenti i dati rilevati:
\begin{itemize}
	\item \verb|Echoes::Sonar|, sull'argomento \verb|sonar_data|, contenenti i dati ricevuti dai sonar (valori delle distanze dei quattro sonar montati sul robot, espresse in millimetri). 
	\item \verb|Echoes::Rfid|, sull'argomento \verb|rfid_data|, contenenti semplicemente una stringa identificativa del tag RFID rilevato
	\item \verb|Echoes::Towers|, sull'argomento \verb|towers_data|, contenente lo stato della torre (abbattuta/non abbattuta) e il numero di fabbriche che sono state abbattute dall'inizio del gioco. %TODO il reset!!!
\end{itemize}
Inoltre espone un servizio di tipo \verb|Echoes::Led| che permette di controllare l'accensione dei led (in particolare, il \emph{numero} di led rossi che si vogliono accendere, e l'accensione di ogni singolo led per quanto riguarda i led gialli e il led verde montato sulla testa di Spykee). %TODO AGGIUNGERE RFID, APRICANCELLI E SISTEMARE LED!!!

A causa dell'alto rumore presente nei valori provenienti dai sonar, il valore che viene pubblicato sul topic \verb|sonar_data| è filtrato attraverso un filtro a media mobile esponenziale. Questo significa che il dato pubblicato all'arrivo del $k$-esimo campione $x_{cur}$ registrato dal sonar è dato da %TODO sicuri che alpha = 0.3?!?
  \[ x_k = \alpha x_{cur} + (1 - \alpha) x_{k-1} \]
dove $\alpha = 0.3$, valore sufficientemente alto da rendere l'aggiornamento dei sonar abbastanza veloce per poter evitare efficacemente l'ostacolo, e sufficientemente basso da ridurre il rumore ad alta frequenza presente nel segnale.

\section{Vision: identificazione degli oggetti}

Il nodo Vision si occupa di analizzare le immagini provenienti dalla telecamera di Spykee, al fine di rilevare la presenza di una torre o di una fabbrica all'interno dell'immagine, mediante riconoscimento dei colori. Riceve da \emph{SpyKee} le immagini della telecamera ed effettua l'identificazione degli oggetti di interesse per il gioco contenuti nell'immagine (la torre e le fabbriche). Analizzata l'immagine, il nodo pubblica un messaggio di tipo \verb|Vision::Results| sull'argomento \verb|vision_results| contenente i dati riguardanti gli oggetti trovati e la loro posizione rispetto al centro dell'immagine. %TODO DETTAGLIARE MESSAGGI

Il cuore del nodo è un algoritmo, sviluppato all'interno del framework MRT \cite{mrt} e già utilizzato in \cite{docmandelli} (a cui si rimanda per ulteriori dettagli), che si occupa di identificare all'interno dell'immagine fornita, dei blob di colore ``simile'' a quello degli oggetti che si stanno cercando. Poiché l'analisi che viene effettuata per identificare le regioni di interesse si basa su un algoritmo di tipo KNN, è necessaria una prima fase di addestramento, che permette di ottenere un classificatore (un file \verb|.kcc|), mediante il quale è possibile associare ad ogni pixel dell'immagine, ossia ad ogni valore $(r,g,b)$, una classe (``torre'', ``fabbrica'' o ``niente'').

Il classificatore viene generato a partire da un file che contiene semplicemente un elenco di valori \verb|BGR| di pixel che si considerano del colore cercato. Per generare questo file è possibile utilizzare il programma \emph{LittleEndian}, evidenziando nell'immagine ricevuta dal robot le aree che corrispondono agli oggetti cercati. È necessario prestare particolare attenzione nel training del classificatore, in quanto è una fase critica per il corretto funzionamento del gioco e il corretto riconoscimento degli oggetti. %% TODO TODO STA ROBA E' DA SISTEMARE!!!

\section{IsAac: il comportamento del robot}
IsAac si occupa di controllare il comportamento di basso livello del robot durante il gioco. Questo nodo ignora le problematiche relative al conteggio dei punti, all'avvio e alla conclusione della partita, e gestisce semplicemente il comportamento del robot (i set-point per i cingoli e l'accensione dei led presenti sul robot) in base alle regole del gioco e ai dati provenienti dai sensori. 

 \emph{IsAac} riceve i messaggi pubblicati da Echoes e Vision, e invia comandi a SpyKee (messaggi \verb|Spykee::Motion| sull'argomento \verb|spykee_motion|) riguardanti il controllo dei cingoli. Inoltre invoca il servizio \verb|led_data| esposto da Echoes per il controllo dei led posti sul robot.

%TODO TODO DA AGGIUNGERE SPIEGAZIONI sintetiche SU BRIAN,FUZZY etc

TODO ...

%TODO espandere?
I comportamenti che sono stati implementati riguardano:
\begin{itemize}
 \item Cercare la torre o le fabbriche (casualmente)
 \item Raggiungere la torre o la fabbrica (una volta che è stata trovata)
 \item Evitare gli ostacoli
 \item Distruggere la torre (o una fabbrica)
\end{itemize}

%TODO dettagliare meglio le regole o il loro funzionamento, spiegare la suddivisione in livelli???

In aggiunta a quanto gestito mediante le regole fuzzy, il nodo si occupa di gestire i comportamenti ``speciali'' del robot, che vengono attivati quando il robot si avvicina ai tag RFID. L'associazione tra i comportamenti e gli identificatori dei corrispondenti tag RFID è definita in un apposito file di configurazione (\verb|rfidconfig.txt|). I comportamenti disponibili, che consistono nella variazione di alcuni dati provenienti dai sensori o diretti agli attuatori per per una durata predefinita, sono:
\begin{itemize}
 \item blocco totale del robot (\verb|lock_all|)
 \item blocco di uno dei due cingoli (\verb|force_rotate_left| e \verb|force_rotate_right|)
 \item blocco della visione (\verb|disable_vision|)
\end{itemize}

%TODO interfaccia grafica? altro?

\section{RoboTower\_Game: la logica di gioco}
Il nodo \emph{RoboTower\_Game} gestisce sia la comunicazione con l'utente mediante un'apposita interfaccia grafica, che la logica ad alto livello del gioco (avvio e arresto delle partite, conteggio dei punti, ...). Interagisce con gli altri modi
\begin{itemize}
\item avviando e fermando IsAac
\item comunicando a IsAac la riabilitazione di eventuali trappole (tag RFID)
\item ricevendo da Echoes le informazioni riguardanti le torri e le fabbriche abbattute
\end{itemize}

%TODO espandere? screenshot?